\documentclass{article}
\usepackage[english]{babel}
\usepackage[utf8x]{inputenc}
\usepackage{tikz}
\usepackage{blindtext}
\usetikzlibrary{arrows}
\usepackage{amsthm}
\usepackage{etoolbox}
\usepackage{setspace}
\usepackage{lipsum}
\AtBeginEnvironment{quote}{\singlespacing\small}


\tikzset{
    vertex/.style={circle,draw,minimum size=1.5em},
    edge/.style={->,> = latex'}
}

\title{Algorithmic transparency in legal matters}
\author{Martin Nørskov \\ Thesis \\ University of Copenhagen }
\begin{document}
\maketitle

\newtheorem{definition}{Definition}

\section*{Abstract}

One possible definition of xxx




\section*{Algorithmic decisions}

Algorithms are increasingly being used in decision-making, as sole decision-makers or as aids for human decision-makers\footnote{At least in the western world.}. The application areas include algorithmic trading, in which algorithms makes decisions about buying or selling assets and credit-scoring customers, credit-scoring of customers and medical decisions. In some countries, including USA and the UK, algorithmic systems are also being deployed in law enforcement and the legal system. 
\\
\\
On the political level several problems with the use of algorithms have been discussed both in Europe (HLEGAI 20xx) and America. In a presentation given to a faction of the U.S. Democratic Party, several "key ethical dimensions" are identified, including "fairness and bias", "opacity and transparency", "trust and expertise" and a "lack of standard for auditing" (D\&S 2016, p. 9)\footnote {An Algorithmic Accountability Act has even been proposed in the House, but not yet passed.}
\\
\\
\section*{Transparency and opacity}
An algorithm is sometimes said to be transparent, or having transparency, when it is known, how the algorithm makes its decisions. When this is not the case, an algorithm is said to be opaque, or having opacity.

Burrell (2016) discerns three different kinds of opacity.
\\
\\
Opacity can be an "intentional corporate or state secrecy", due to "technical illiteracy" or stemming from characteristics and scale of algorithmic systems.
\\
\\
Corporations and governments can have a number of motives for such secrecy, including reasons of competition and intellectual property and security through obscurity. Another reason is that the algorithm might only be able to function properly under opacity. If its inner workings are transparent, people may be able to "game" it, as seen in Search Engine Optimization (SEO), in which website-owners are trying to optimize their placing in Google search results.

Opacity
\newpage

\section*{Conclusion: xy}
xxxx

\newpage
\begin{thebibliography}{1}

\bibitem{Boghossian2006}
Boghossian, Paul (2006).
\newblock {\em Fear of knowledge.}
\newblock Oxford: Oxford UP

\bibitem{Carter2017}
Carter, A. (2017).
\newblock  {Epistemic pluralism, epistemic relativism and "Hinge" epistemology}
\newblock In Coliva, A. et al. (2017). {\em Epistemic pluralism.}
\newblock London: Palgrave Macmillan

\bibitem{Goldman2001}
Goldman, Alvin(2001).
\newblock  {Experts: Which Ones Should you trust?}
\newblock {\em Philosophy and Phenomenological Research}, Volume 63, 2001, Juli - No. 1




\end{thebibliography}
\end{document}