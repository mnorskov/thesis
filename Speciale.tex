\documentclass{article}
\usepackage[english]{babel}
\usepackage[utf8x]{inputenc}
\usepackage{tikz}
\usepackage{blindtext}
\usetikzlibrary{arrows}
\usepackage{amsthm}
\usepackage{etoolbox}
\usepackage{setspace}
\usepackage{lipsum}
\AtBeginEnvironment{quote}{\singlespacing\small}


\tikzset{
    vertex/.style={circle,draw,minimum size=1.5em},
    edge/.style={->,> = latex'}
}

\title{Algorithmic transparency in legal matters}
\author{Martin Nørskov \\ Thesis \\ University of Copenhagen }
\begin{document}
\maketitle

\newtheorem{definition}{Definition}

\section*{Abstract}

One possible definition of xxx




\section*{Algorithmic decision}
Algorithms are increasingly being used in decision-making, as sole decision-makers or as aids for human decision-makers.
\\
\\
\textbf{xx}

xx

\newpage

\section*{Conclusion: xy}
xxxx

\newpage
\begin{thebibliography}{1}

\bibitem{Boghossian2006}
Boghossian, Paul (2006).
\newblock {\em Fear of knowledge.}
\newblock Oxford: Oxford UP

\bibitem{Carter2017}
Carter, A. (2017).
\newblock  {Epistemic pluralism, epistemic relativism and "Hinge" epistemology}
\newblock In Coliva, A. et al. (2017). {\em Epistemic pluralism.}
\newblock London: Palgrave Macmillan

\bibitem{Goldman2001}
Goldman, Alvin(2001).
\newblock  {Experts: Which Ones Should you trust?}
\newblock {\em Philosophy and Phenomenological Research}, Volume 63, 2001, Juli - No. 1




\end{thebibliography}
\end{document}